\documentclass[tikz,border=2mm]{standalone}

% =================================================
% Parameters (overridden externally)
% =================================================

\providecommand{\metric}{mse}
\providecommand{\YLABEL}{}

\def\YMIN{0.5}
\def\YMAX{2}
% =================================================
% Packages
% =================================================
\usepackage{pgfplots}
\usepackage{pgfplotstable}
\usepgfplotslibrary{groupplots}
\pgfplotsset{compat=1.18}
\usepackage{xcolor}

% =================================================
% Paths
% =================================================
\def\tabledir{../../output/secondary/tables}

% =================================================
% Axis style (same spirit as main figures)
% =================================================
\pgfplotsset{
  performanceaxis/.style={
    width=7.5cm,
    height=6cm,
    axis lines=left,
    axis line shift=4pt,
    xmin=0,
    xmax=0.7,
    ymin=\YMIN,
    ymax=\YMAX,
    xlabel={Observed missingness proportion},
    ylabel={\YLABEL},
    tick label style={font=\small},
    label style={font=\small},
    title style={font=\small},
    grid=none
  }
}

% =================================================
% Plot styles
% =================================================
\pgfplotsset{
  every axis plot/.append style={line width=1.2pt},
  points/.style={only marks, mark=*, mark size=0.55pt, opacity=0.15},
  loess/.style={very thick}
}

% =================================================
% Colors (secondary analysis)
% =================================================

% --- MLE family (blue tones)
\definecolor{mleall}{RGB}{60,179,113}        % steel blue
\definecolor{miall}{RGB}{34,139,34}   % darker blue

% --- MI family (green tones)
\definecolor{mleobservedY}{RGB}{70,130,180}         % medium sea green
\definecolor{miobservedY}{RGB}{30,90,150}    % forest green

% --- References
\definecolor{refMU}{RGB}{0,0,139}             % dark blue
\definecolor{refMC}{RGB}{139,0,0}             % dark red

% =================================================
% One subgroup plot (Scenario 5)
% =================================================
\newcommand{\secondarysubplot}[1]{%
  % #1 = overall / complete / incomplete

  % mleall
  \addplot[mleall, points, forget plot]
    table[col sep=comma, ignore chars={"},
          x=observedmissingness, y=mleall]
    {\tabledir/M5_#1_\metric_points.csv};
  \addplot[mleall, loess]
    table[col sep=comma, ignore chars={"},
          y=mleall]
    {\tabledir/M5_#1_\metric_loess.csv};

  % mleobservedY
  \addplot[mleobservedY, points, forget plot]
    table[col sep=comma, ignore chars={"},
          x=observedmissingness, y=mleobservedY]
    {\tabledir/M5_#1_\metric_points.csv};
  \addplot[mleobservedY, loess]
    table[col sep=comma, ignore chars={"},
          y=mleobservedY]
    {\tabledir/M5_#1_\metric_loess.csv};

  % miall
  \addplot[miall, points, forget plot]
    table[col sep=comma, ignore chars={"},
          x=observedmissingness, y=miall]
    {\tabledir/M5_#1_\metric_points.csv};
  \addplot[miall, loess]
    table[col sep=comma, ignore chars={"},
          y=miall]
    {\tabledir/M5_#1_\metric_loess.csv};

  % miobservedY
  \addplot[miobservedY, points, forget plot]
    table[col sep=comma, ignore chars={"},
          x=observedmissingness, y=miobservedY]
    {\tabledir/M5_#1_\metric_points.csv};
  \addplot[miobservedY, loess]
    table[col sep=comma, ignore chars={"},
          y=miobservedY]
    {\tabledir/M5_#1_\metric_loess.csv};

  % refMU
  \addplot[refMU, loess, dashed]
    table[col sep=comma, ignore chars={"},
          y=refMU]
    {\tabledir/M5_#1_\metric_loess.csv};

  % refMC
  \addplot[refMC, loess, dashed]
    table[col sep=comma, ignore chars={"},
          y=refMC]
    {\tabledir/M5_#1_\metric_loess.csv};
}

% =================================================
% Document
% =================================================
\begin{document}

\begin{tikzpicture}

\begin{groupplot}[
  group style={
    group size=2 by 2,
    horizontal sep=2cm,
    vertical sep=2cm
  },
  performanceaxis
]

% =================================================
% Legend cell (top-left)
% =================================================
\nextgroupplot[
  axis lines=none,
  ticks=none,
  xmin=0, xmax=1,
  ymin=0, ymax=1,
  legend to name=legendSecondary,
  legend style={draw=none}
]

\addplot[mleall, loess] coordinates {(0,0)};
\addlegendentry{MLE (all)}

\addplot[mleobservedY, loess] coordinates {(0,0)};
\addlegendentry{MLE (observed Y)}

\addplot[miall, loess] coordinates {(0,0)};
\addlegendentry{MI (all)}

\addplot[miobservedY, loess] coordinates {(0,0)};
\addlegendentry{MI (observed Y)}

\addplot[refMU, loess, dashed] coordinates {(0,0)};
\addlegendentry{Optimal MU}

\addplot[refMC, loess, dashed] coordinates {(0,0)};
\addlegendentry{Optimal MC}

% =================================================
% Overall
% =================================================
\nextgroupplot[title={Overall}, ymin=0, ymax=2]
\secondarysubplot{overall}

% =================================================
% Complete
% =================================================
\nextgroupplot[title={Complete}, ymin=0, ymax=5]
\secondarysubplot{complete}

% =================================================
% Incomplete
% =================================================
\nextgroupplot[title={Incomplete}, ymin=0, ymax=5]
\secondarysubplot{incomplete}

\end{groupplot}

% Legend placement (centered in top-left cell)
\node[anchor=center]
  at ([xshift=3.75cm,yshift=-3cm]current bounding box.north west)
  {\pgfplotslegendfromname{legendSecondary}};

\end{tikzpicture}

\end{document}

